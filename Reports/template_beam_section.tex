\documentclass[11pt, a4paper]{report}
\usepackage{graphicx}
\usepackage[utf8]{inputenc}
\usepackage{amssymb}
\usepackage{multicol}
\usepackage{multirow}
\usepackage{caption}
\usepackage{mathtools}

\graphicspath{ {./Images} }


\title{Diseño de viga rectangular}

\begin{document}

\begin{center}
\centering 
\textbf{Memoria de cálculo de diseño de viga a flexión}
\vspace{0.3cm}

\textbf{Método de compatibilidad de deformaciones}
\vspace{0.2cm}
\end{center}

\begin{enumerate}  
    \item Distribución de aceros propuesta
    \begin{center}
    \begin{minipage}{8.2cm} 
        \centering 
        \setlength\fboxsep{0pt}
        \setlength\fboxrule{0.3pt}

        \includegraphics[trim={1mm 0.5mm 1mm 0.5mm},clip,width=1.0\textwidth]{Beam Section.png}
    \end{minipage}
    \end{center}

    \[f'c = {{beam.section.fc}}kg/cm^2\]
    \[fy = {{beam.section.fy}}kg/cm^2\]
    
    \item Condición de falla: $e_t>=0.004$
    \item Compatibilidad de deformaciones:
    %%Section.plot_comp_defo and stresses distribution (put it in just one figure with two axis)
    
    \begin{figure}[h]  %%It is importat to place the figure in a good place
        \begin{multicols}{2}
            \begin{minipage}[c]{6.5cm}
                \centering
                \setlength\fboxsep{0pt}
                \setlength\fboxrule{0.3pt}
                \includegraphics[trim={1mm 0.5mm 1mm 0.5mm},clip,width=1.0\textwidth]{Beam sect.png}
            \end{minipage}
            \captionsetup{justification=centering}
            \captionof{figure}{Sección}

            \begin{minipage}[c]{6.5cm}
                \centering
                \setlength\fboxsep{0pt}
                \setlength\fboxrule{0.3pt}
                \includegraphics[trim={1mm 0.5mm 1mm 0.5mm},clip,width=1.0\textwidth]{Beam deformation compatibility.png}
            \end{minipage}
            \captionsetup{justification=centering}
            \captionof{figure}{Deformaciones}
        \end{multicols}
    \end{figure}

    \item Cálculos:

    \begin{enumerate}
    
    \item Cálculos previos
    \[ c={{ beam.section.c|round(1) }} cm \]
    \[ \beta _1={{beam.section.b1|round(1)}} \rightarrow a=\beta _1 c = {{beam.section.a|round(1)}} cm \]
    \[ Aw = {{ beam.section.Aw|round(1) }} cm^2 \] %corresponds to the Witney section of the beam section
    \item Fuerzas de acero y concreto
    \[ C_c = 0.85f'c Aw = {{beam.section.Cc|round(1)}} tf-m\]

    \begin{equation}
    \begin{dcases}
    
        es_{{i+1}}={{beam.section.es[i]|round(4)}} \rightarrow fs_{{i+1}}={{beam.section.fs[i]|round|int}} kgf/cm^2 \rightarrow Fs_{{i+1}}=As_{{i+1}}fs_{{i+1}}={{beam.section.Fs[i]|round(2)}} tf \\
    
    \end{dcases}
    \end{equation}

    \item Fuerzas nominales
    \[ Pn = C_c + \sum Fs_i = {{beam.section.Pn|round(1)}} tonf\]
    \[ Mn = C_c(CP-\bar{y_c}) + \sum Fs_i(CP-y_i) = {{beam.section.Mn|round(1)}} tonf-m\]
    
    \item Tipo de falla:
    
    

    
        $ et={{et}} < 0.005 \therefore \text{Falla ductil} $
    
        $ et={{et}} < {{ey}} \therefore \text{Falla frágil} $ 
    
        $ ey={{ey}} < et={{et}}<0.005 \therefore \text{Falla intermedia} $
    

    \item Acero mínimo según E060:
    $$ As_{min} = 0.7\frac{\sqrt{f'c}}{fy}b_w d = {{beam.Asmin|round(1)}} cm^2 $$
    $$ As_{tot} = {{beam.As_tot|round(1)}} cm^2 $$
    
        $$ As_{total} \geq As_{min} \therefore \text{Cumple con la cuantía mínima} $$
    
        $$ As_{total}<As_{min} \therefore \text{No cumple con la cuantía mínima} $$
    

    \end{enumerate}
\end{enumerate}
\end{document}